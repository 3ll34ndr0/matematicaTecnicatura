\documentclass[a4paper, twocolumn]{article}
\usepackage[a4paper,top=0.5in, bottom=0.75in, left=0.4in, right=0.3in]{geometry}
\usepackage[utf8]{inputenc}
\usepackage[spanish]{babel}
\usepackage{ifpdf}
\usepackage{hyperref}
\usepackage{booktabs}
\usepackage{amsmath}
\usepackage{amsfonts}
\usepackage{enumerate}
\usepackage{verbatim}
\usepackage{tikz}

\title{Resultados Unidad 0}
%\author{}
\date{Febrero 2017}
\ifpdf
\hypersetup{
    pdfauthor={Leandro Marsó},
    pdftitle={Ejercicios resueltos unidad 0},
}
\fi



%%%%
% Para los conjuntos, defino unas figuras que uso frecuentemente:
\def\circuloN{( 30:0.8) circle (1.4)}
\def\circuloM{(270:0.8) circle (1.4)}
\def\circuloP{(150:0.8) circle (1.4)}
%%%%%

\begin{document}
\maketitle
\subsubsection*{Actividad 4}
\begin{enumerate}[1)]
\item{F}
\item{V}
\item{F}
\item{V}
\item{F}
\item{V}
\item{F}
\item{F}
\end{enumerate}
\subsubsection*{Actividad 5}
% Please add the following required packages to your document preamble:
% \usepackage{booktabs}
\begin{table}[h]
\centering

\label{my-label}
\begin{tabular}{@{}cccccc@{}}
\toprule
a    & b  & a-b  & ab   & a:b & -(a+b) \\ \midrule
4    & -2 & 6    & -8   & -2  & -2     \\
0    & -5 & 5    & 0    & 0   & 5      \\
-9   & -3 & -6   & 27   & 3   & 12     \\
5    & -5 & 4    & -25  & -1  & 0      \\
-12  & 6  & -18  & -72  & -2  & 6      \\
-256 & -8 & -248 & 2048 & 32  & 264    \\
35   & -7 & 42   & -245 & -5  & -28   
\end{tabular}
\end{table}



\subsubsection*{Actividad 7}
\begin{enumerate}[a)]
\item $\frac{18}{48}$
\item $\frac{39}{104}$
\item $\frac{12}{32}$
\item $\frac{15}{40}$
\end{enumerate}

\subsubsection*{Actividad 8}
\begin{enumerate}[a)]
\item $\frac{3}{4} = \frac{21}{28} = \frac{15}{20}$
\item $\frac{40}{25} = \frac{120}{75} = \frac{8}{5}$
\item $\frac{9}{12} = \frac{81}{108} = \frac{3}{4}$
\item $\frac{5}{2} = \frac{250}{100} = \frac{40}{16}$
\item $\frac{-1}{95} = \frac{-3}{285} = \frac{-5}{475}$
\item $\frac{2}{7} = \frac{14}{49} = \frac{24}{84}$
\end{enumerate}
\subsubsection*{Actividad 9}
\begin{enumerate}[a)]
\item $\frac{11}{9} > \frac{13}{11}$
\item $\frac{3}{5} > \frac{4}{7}$
\item $\frac{3}{8} < \frac{5}{11}$
\item $\frac{-1}{9} < \frac{13}{5}$
\item $\frac{-5}{7} > \frac{-3}{2}$
\item $\frac{15}{2} < \frac{23}{3}$
\item $\frac{-1}{7} < \frac{-1}{9}$
\item $\frac{9}{13} > \frac{8}{17}$
\item $\frac{1}{3} = \frac{2}{6}$
\item $\frac{-3}{8} < \frac{3}{5}$
\item $\frac{3}{8} > \frac{3}{4}$
\item $\frac{9}{2} = \frac{18}{4}$
\end{enumerate}


\subsubsection*{Actividad 10}
\begin{enumerate}[a)]
\item $\frac{17}{5}=3.4$
\item $\frac{32}{3}=10.\overline{6}$
\item $\frac{2}{45}=0.0\overline{4}$
\item $\frac{11}{15}=0.7\overline{3}$
\item $\frac{101}{330}=0.3\overline{06}$
\end{enumerate}

\subsubsection*{Actividad 11}
\begin{enumerate}[1)]
\item $0.\overline{8} = \frac{8}{9}$
\item $2.\overline{5} = \frac{23}{9}$
\item $3.\overline{15} = \frac{104}{33}$
\item $0.\overline{18} = \frac{2}{11}$
\item $2.5\overline{2} = \frac{227}{90}$
\item $5.2\overline{4} = \frac{472}{90}$
\end{enumerate}

\subsubsection*{Actividad 12}
\begin{enumerate}[a)]
\item $\frac{256}{390625}$
\item $\left(\frac{3}{8}\right )^8$
\item $\frac{16}{25}$
\item $\frac{4}{9}$
\item $\frac{729}{15625}$
\item $1$
\item $\frac{7}{4}$
\item $1$
\item $\frac{27}{8}$
\item $16$
\item $\frac{125}{8}$
\item $\frac{1}{32}$
\item $2$
\item $1$
\item $2$
\end{enumerate}

\subsubsection*{Actividad 13}
\begin{enumerate}[a)]
\item $\neq$, la raíz no distribuye sobre la suma.
\item $=$
\item $=$
\item $\neq$, la potencia no distribuye sobre la suma.
\item $=$
\item $=$
\item $\neq$, la potencia distribuye sobre el producto.
\item $\neq$, la potencia distribuye sobre el producto.
\end{enumerate}
\subsubsection*{Actividad 14}
\begin{enumerate}[a)]
\item $a$
\item $x^{11}$
\item $\frac{1}{b}$
\item $a^{15}$
\item $\frac{1}{a^7b^2}$
\item $b^4$

\end{enumerate}

\subsubsection*{Actividad 15}
\begin{enumerate}[a)]
\item $4$
\item ${2^{-36}}$
\item $27$
\item $\frac{1}{4}$
\item $-6$
\item $\frac{5}{4}$
\item $\frac{2}{3}$
\item $2$
\end{enumerate}

\subsubsection*{Actividad 16}
\begin{enumerate}[a)]
\item $\frac{-1}{2}$
\item $\frac{25}{9}$
\item $\frac{10}{3}$
\item $\frac{-64}{27}$
\item $\frac{-27}{50}$
\item $\frac{34}{9}$
\item $\frac{-41}{30}$
\item $\frac{-7}{25}$
\end{enumerate}

\subsubsection*{Actividad 17}
\begin{enumerate}[a)]
\item $\frac{61}{90}$
\item $\frac{5}{24}$
\item $\frac{11}{2}$
\item $-\frac{31}{75}$
\end{enumerate}

\subsubsection*{Actividad 18}
\begin{enumerate}[a)]
\item $4\sqrt{10}$
\item $6\sqrt{5}$
\item $10\sqrt{3}$
\item $5\sqrt[3]{2}$
\item $\sqrt[3]{3}$
\item $4a^2b^2c^3\sqrt[3]{2ab}$
\item $\frac{ac}{bd^2}\sqrt[5]{2\frac{a^2c^4}{b^3}}$
\item $10a^3 d^2\sqrt{a}$
\end{enumerate}

\subsubsection*{Actividad 19}
\begin{enumerate}[a)]
\item $\frac{7}{2}\sqrt{5}$
\item $-3\sqrt{2}$
\item $\frac{13}{3}\sqrt[3]{5}$
\item $-\frac{1}{3}\sqrt{\frac{5}{2}}$
\item $-\sqrt{5}$
\item $4\sqrt[3]{5}$
\item $\sqrt[3]{2}$
\item $13\sqrt[3]{5}$
\item $112\sqrt{13}$
\item $-18\sqrt{2}$
\item $55\sqrt{2}$
\item ...
\item $35\sqrt{5}$
\item $6 + \frac{13}{6}\sqrt{2}$
\item $24\sqrt[3]{5}$
\end{enumerate}

\subsubsection*{Actividad 20}
\begin{enumerate}[a)]
\item $4\sqrt[6]{2\times5^{5}\times3^2}$
\item $80$
\item $20a^3b^4c^4\sqrt[10]{a^9b^4c^7}$
\item $2ab^2c^3\sqrt[12]{\frac{2^{10} a^7 b^5}{3^6 d^3}}$
\item $2\sqrt[6]{2^5\times2^{-6}\times3^{-6}}$
\item $\sqrt[10]{(a-x)^7}$
\item $24\sqrt[10]{2^9}$
\item $\sqrt[12]{{5^5}\times3^{-3}\times2^{-6}}$
\end{enumerate}

\subsubsection*{Actividad 21}
\begin{enumerate}[a)]
\item $\frac{3\sqrt{5}+\sqrt{10}}{10}$
\item $2\sqrt[5]{2}$ %Corregir
\item $6+4\sqrt{3}$
\item $\frac{\sqrt{\pi}}{\pi}$
\item $\frac{7-2\sqrt{10}}{3}$
\item $\frac{-2\sqrt[4]{a^2}}{a}$ 
\item $\frac{2}{5}\sqrt[4]{5^3}$
\item $\sqrt[7]{a^3}$
\item $\frac{\sqrt[6]{2}}{2}$
\item $3-\sqrt{6}-2\sqrt{3}+2\sqrt{2}$
\item $\sqrt{7}-2$
\end{enumerate}

\subsubsection*{Actividad 22}
\begin{enumerate}[a)]
\item $6$
\item $3$
\item $3$
\item $-2$
\item $-3$
\item $-3$
\item $\frac{1}{2}$
\item $\frac{2}{3}$
\item $0$
\end{enumerate}

\subsubsection*{Actividad 23}
\begin{enumerate}[a)]
\item $\log_2{1}=0$
\item $\log_4{16}=2$
\item $\log_2{2}=1$
\item $\log_4{1}=0$
\item $\log_2{64}=6$
\item $\log_4{\frac{1}{64}}=-3$
\end{enumerate}


\subsubsection*{Actividad 25}
\begin{enumerate}[a)]
\item $-1,678\dots$
\item $100$
\item $4$
\item $0,0500031$
\item $1,999219017$
\item $1,080482024$
\item $11$
\item $2$
\item $\frac{1}{2}$
\item $6$
\item $4$
\item $0$
\item $1,02219706$
\item $\frac{1}{4}$
\item $1$
\item $0,903089987$
\item $-2$
\item $\pm 1,791385485$
\item $1,940029217$
\item $4$
\end{enumerate}




\subsubsection*{Actividad 26}
\begin{enumerate}[a)]
\item $A=\{$rojo, naranja, amarillo, verde, azul, índigo, violeta$\}$
\item $B=\{$Enero, Febrero, Marzo, Abril, Mayo, \\
 Junio, Julio, Agosto,Septiembre,  \\
Octubre, Noviembre, Diciembre$\}$
\item $C=\{5, 7, 9, 11, 13\}$
\item $D=\{6, 9, 12, 15\}$
\item $E=\{25, 30, 35\}$
\item $F=\{$Acuario, Leo, Sagitario, Virgo,
 Cancer, \\
Capricornio, Tauro, Géminis, Aries, Escorpio, \\
Libra, Piscis$\}$
\item $G=\{2,3,5,7,11,13\}$ 
\end{enumerate}

\subsubsection*{Actividad 27}
\begin{enumerate}[a)]
\item $\text{A}=\{x \in \mathbb{Z} / x\leq 9 \}$
\item $\text{B}=\{x \in \mathbb{N} / x \mbox{ es impar} \wedge x \leq 9 \}$

\item $\text{C}=\{x / x \mbox{ es color primario}\}$

\item $\text{C}=\{x / x \mbox{ es dedo de la mano}\}$

\item $\text{C}=\{x / x \mbox{ es planeta del sistema solar}\}$


\end{enumerate}

\subsubsection*{Actividad 28}
\begin{enumerate}[a)]
\item V
\item F
\item F

\end{enumerate}

\subsubsection*{Actividad 29}
\begin{enumerate}[a)]
\item V
\item F
\item F
\item F
\end{enumerate}

\subsubsection*{Actividad 30}
\begin{enumerate}[a)]
\item Finito
\item Unitario
\item Infinito
\item Unitario
\item Vacío
\item Vacío
\end{enumerate}

\subsubsection*{Actividad 31}
\begin{enumerate}[a)]
\item $A=\{0, 1, 5, 6, 7, 8, 9\}$
\item $B=\{\mbox{e, i, o}\}$
\item $C=\{\mbox{naranja}, \mbox{verde}, \mbox{índigo}, \mbox{violeta}\}$
\end{enumerate}

\subsubsection*{Actividad 32}
\begin{enumerate}[a)]
\item $A \cup B =\{1,2,3,4,5,6,8\}$
\item $A \cup C =\{3,4,5,7,8,9\}$
\item $B \cup C =\{1,2,3,4,5,6,7,9\}$
\item $A \times B = \{(3,1) ; (3,2 ) ; (3,3 ) ; (3,4 ) ; (3,6 ) ; (5,1 ) ; (5,2 ) ;(5,3 ) ;\\
(5,4 ) ; (5,6 ) ; (4,1 ) ; (4,2 ) ; (4,3 ) ; (4,4 ) ; (4,6 ) ; (8,1 ) ; (8.2 ) ; (8,3 ) ; \\
(8,4 ) ; (8,6 )\}$
\item $C \times B = \{(5,1 ) ; (5,2 ) ; (5,3 ) ; (5,4 ) ; (5,6 ) ; (7,1 ) ; (7,2 ) ; (7,3 ) ;\\
  (7,6 ) ; (7,4 ) ;(9,1 ) ; (9,2 ) ; (9,3 ) ; (9,4 ) ;(9,6)\}$
\end{enumerate}


\subsubsection*{Actividad 33}
\begin{enumerate}[a)]
\item $\textnormal{M} \cap \textnormal{N} =\{5, 7\}$ \\
\begin{tikzpicture}
%Esto es para colorear el resultado (una forma de hacerlo):
  \begin{scope}
  \clip ( 0:0.8) circle (1.4);
  \fill[lightgray] (180:0.8) circle (1.4);
  \end{scope}
%%%%%%%%%%%%%%%%%

%Dibujo los círculos: sintáxis:
%(rotación: radio de desplazamiento) figura (tamaño de la misma)
    \draw[black]   ( 0:0.8) circle (1.4);
    \draw[black] (180:0.8) circle (1.4);
 
 % Ubico los elementos
  \node at ( 180:1.3)   {8};
  \node at ( 140:1.4)   {0};
  \node at ( 220:1.4)   {2};
  \node at (135:2.15) {\emph{M}};
  \node at ( 330:1.2)   {6};
  \node at ( 25:1.4)   {9};
  \node[align=center] at ( 45:2.15) {\emph{N}};
  \node[align=center] {5 \\ \\7 };
\end{tikzpicture}

\item $\textnormal{M} \cap \textnormal{N} \cap \textnormal{P} =\{5\}$ \\ \\
\begin{tikzpicture}
%Esto es para colorear algún subconjunto:
  \begin{scope}
  \clip \circuloN;
  \clip \circuloM;
  \fill[lightgray] \circuloP;
  \end{scope}
%%%%%%%%%%%%%%%%%


  \draw[black]   ( 30:0.8) circle (1.4);
  \draw[black] (150:0.8) circle (1.4);
  \draw[black] (270:0.8) circle (1.4);
 
  \node at (270:2.5) {\emph{M}};
  \node[align=center] at ( 30:2.5) {\emph{N}};
  \node[align=center] at ( 150:2.5) {\emph{P}};
  
  \node at ( 185:0.9)   {8};
  \node at ( 225:0.9)   {0};
  \node at ( 150:1.4)   {1};
  \node at ( 210:1.3)   {2};
  \node at ( 90:1.)   {6};
  \node at ( 35:1.4)   {9};
  \node {5};
  \node at (-25:1){ 7 };
\end{tikzpicture}
\item 
\item $\textnormal{M} - \textnormal{N} = \{7\}$ \\
\begin{tikzpicture}
%Coloreo:
  \begin{scope}
  \clip (-2,-2) rectangle (2.5,2) (180:0.8) circle (1.4);
  \fill[lightgray] (0:0.8) circle (1.4) ;
  \end{scope}

%Dibujo los círculos: sintáxis:
%(rotación: radio de desplazamiento) figura (tamaño de la misma)
    \draw[black]   ( 0:0.8) circle (1.4);
    \draw[black] (180:0.8) circle (1.4);
 
 % Ubico los elementos
  \node at ( 150:1.3)   {1};
  \node at ( 200:1.4)   {6};
  \node at (135:2.15) {\emph{P}};
  \node at ( 360:1.2)   {7};
  \node[align=center] at ( 45:2.15) {\emph{M}};
  \node[align=center] {0 \\5 \\ 2 \\8 };
\end{tikzpicture}

\item $\textnormal{M} \cap (\textnormal{P} - \textnormal{N}) = \{0,2,8\}$ 

\begin{tikzpicture}
%Esto es para colorear algún subconjunto:
  \begin{scope}
  \clip (-2.5,-2) rectangle (2.5,2) \circuloN;
  \clip \circuloM \circuloN;
  \fill[lightgray] \circuloP ;
  \end{scope}
%%%%%%%%%%%%%%%%%

  \draw[black]   \circuloN;
  \draw[black] \circuloP;
  \draw[black] \circuloM;

  \node at (270:2.5) {\emph{M}};
  \node[align=center] at ( 30:2.5) {\emph{N}};
  \node[align=center] at ( 150:2.5) {\emph{P}};
  
  \node at ( 185:0.9)   {8};
  \node at ( 225:0.9)   {0};
  \node at ( 150:1.4)   {1};
  \node at ( 210:1.3)   {2};
  \node at ( 90:1.)   {6};
  \node at ( 35:1.4)   {9};
  \node {5};
  \node at (-25:1){ 7 };
\end{tikzpicture}
\end{enumerate}


\subsubsection*{Actividad 34}
\begin{enumerate}[a)]
\item Hacemos un diagrama de Venn de todos los elementos.

\begin{tikzpicture}

    \draw[black]   ( 90:1.2) circle (2);
    \draw[black] (210:1.2) circle (2);
    \draw[black]  (330:1.2) circle (2);
    \draw[black] (-4,-3) rectangle (4.3,4) node [text=black,above] {$U$};
  
%(-2,-2) rectangle (3,2) node [text=black,above] {$H$};

  \node at ( 90:2)    {9};
  \node at ( 90:3.5)    {Francés};
  \node at ( 150:1.5)   {17};
  \node at ( 210:2)   {29};
  \node at (210:3.8) {Inglés};
  \node at ( 330:2)   {9};
  \node at (330:3.8) {Alemán};
  \node at ( 25:1.3)   {11};
  \node [font=\Large] {8};
  \node at (50:4.5) {23};
\end{tikzpicture}

Leen sólo inglés: 29. Lee un sólo idioma: 47. Ningún idioma: 23. Sólo francés: 9. Sólo alemán: 9.
\item Realizamos un diagrama para visualizar los elementos:\\ \\
\begin{tikzpicture}
    \draw[black]   ( 0:1.2) circle (2);
    \draw[black] (180:1.2) circle (2);
    %(-4,-2.5) rectangle (5,3) node [text=black,above] {$H$};
  
  \node at ( 150:2)   {Pedro};
  \node at ( 210:2)   {Hugo};
  \node at ( 180:2)   {Roberto};
  \node at (145:3.8) {\emph{El {P}ichi}};
  \node at ( 330:2)   {Andrea};
  \node at ( 25:2)   {José};
  \node[align=center] at ( 46:3) {\emph{Olimpiadas Matemáticas}};
  \node[align=center] {Edgar\\Cristina \\Rolando\\Diego};
\end{tikzpicture}

Están en ambos equipos: Edgar, Rolando, Cristina y Diego. 
%Están en al menos uno de los dos equipos: ????
Están en el equipo de futbol solamente: Pedro, Hugo, Roberto.
Están sólo en uno de los dos equipos: Pedro, Hugo, Roberto, Andrea y José.

\item Luego de realizar el diagrama de Venn, extraemos las siguientes conclusiones: Todos los géneros prestados: Pop, Rock, Punk, Gothic, Clasica, Jazz, Salsa, Hip hop, Metal, Industrial. Géneros a oir primero: Pop, Gothic.

\begin{tikzpicture}
    \draw[black]   ( 0:.85) circle (1.5);
    \draw[black] (180:.85) circle (1.5);
    %(-4,-2.5) rectangle (5,3) node [text=black,above] {$H$};
  
  \node at ( 150:1.5)   {Rock};
  \node at ( 195:1.5)   {Punk};
  \node at ( 180:1.5)   {Clasica};
  \node at ( 165:1.5)   {Jazz};
  \node at (145:3.0) {\emph{Laura}};
  \node at ( 345:1.5)   {Salsa};
  \node at ( 15:1.5)   {Hip hop};
  \node at ( 30:1.5)   {Metal};
  \node at ( 0:1.5)   {Industrial};
  \node[align=center] at (46:2.5) {\emph{Diana}};
  \node[align=center] {Pop\\Gothic};
\end{tikzpicture}



\item Practican natación: 20. Sólo natación: 8. Algún deporte: 40

\begin{tikzpicture}

    \draw[black]   ( 0:.5) circle (1);
    \draw[black] (180:.5) circle (1);
    \draw[black] (-2.5,-1.75) rectangle (2.5,1.75) node [text=black,above] {$U$};
  
%(-2,-2) rectangle (3,2) node [text=black,above] {$H$};

  \node at ( 270:1.3)    {10};
  \node at ( 45:1.75)    {Futbol};
  \node at ( 180:1.0)   {17};
  \node at (135:1.75) {Natación};
  \node at ( 0:1.0)   {8};
  \node  {12};
\end{tikzpicture}

\item 


\begin{tikzpicture}

    \draw[black]   ( 0:.5) circle (1);
    \draw[black] (180:.5) circle (1);
    \draw[black] (-2.5,-1.75) rectangle (2.5,1.75) node [text=black,above] {$U$};
  

  \node at ( 270:1.3)    {20};
  \node at ( 45:1.75)    {Física};
  \node at ( 180:1.0)   {40};
  \node at (135:1.75) {Matemática};
  \node at ( 0:1.0)   {25};
  \node  {12};
\end{tikzpicture}



\end{enumerate}

\pagebreak
\subsubsection*{Actividad 44}
\begin{enumerate}[a)]
\item $5x^3 + \frac{7}{2}x^2 - \frac{7}{2}x$
\item $-\frac{3}{2}a^2 + 5a + \frac{15}{8}b$
\end{enumerate}

\subsubsection*{Actividad 45}
\begin{enumerate}[a)]
\item $-2x^3 + 5x^2 - 12x +5$
\item $16x^7 - 8x^6 +24x^4 - 20x^3 -2x^2$
\item $6y^3 - 3y^2 + 9y + 27$
\item $-15x^6 -6x^5 + \frac{43}{4}x^4-x^3-\frac{5}{2}x^2 + \frac{1}{4}x$
\item $a^4 - 1$
\item $12x^3 -26x^2 + 13x -5$
\end{enumerate}

\subsubsection*{Actividad 46}
\begin{enumerate}[a)]
\item $q(x) = -4x^2 + 3x -5$ , $r(x)=0$
\item $q(x) = 40x^4 -36x^3 -12x -24$ , $r(x)=0$
\item $q(x) = 2x+\frac{2}{3}$ , $r(x)=\frac{19}{3}x - \frac{5}{6}$
\item $q(x) = 2x^3 -2x^2 +\frac{11}{4}x-\frac{27}{8}$ , $r(x) = \frac{27}{4}x -\frac{81}{8}$
\end{enumerate}

\subsubsection*{Actividad 47}
\begin{enumerate}[a)]
\item $q(x) = x - 5$, $r(x) = 0 $
\item $q(x) = x^2 -3x +2$, $r(x) = 0$
\item $q(x) = x^3 - x^2 - x -1$, $r(x)=2$
\item $q(x) = x^2 + 3x + 9$, $r(x)=0$
\item $q(x) = \frac{2}{5}x^2+\frac{1}{5}x+\frac{9}{10}$, $r(x)=0$

\end{enumerate}

\subsubsection*{Actividad 49}
\begin{enumerate}[a)]
\item $\frac{3}{4}x^2 + \frac{13}{2}x+\frac{13}{2}$
\item $-x^3 + 6x^2 -6x +7$
\item $3x + 8$, resto $=25$
\end{enumerate}

\subsubsection*{Actividad 50}
\begin{enumerate}[a)]
\item $5 a^2 (3 a^4 b^2 - 2 a^3 c + 5 a b^2 + b^3)$
\item $4m(n+4p+8mq)$
\item $x^2(x+2-7x^2)$
\item $4(x+5)$
\item $7m(1-2m^4+3m^2)$
\item $7abc^2(a^2b^2c^6-2ac^3+7b^3+4ab^2c)$
\item $\frac{3}{5}yz(xy-2az^2-\frac{3}{4}z^5)$
\end{enumerate}

\subsubsection*{Actividad 51}
\begin{enumerate}[a)]

\item $(a+b)(2x-y+5)$
\item $(a+b-c)(m-n+x)$
\item $(ax-b)(\frac{1}{2}a-2x+1)$
\item $(ab+2m)(2a-3b)$
\item $(3m-x)(5x+2-y)$
\item $(x^3-2y^3)(3x^2+\frac{1}{3}y^2-xy)$
\item $(x-1)(7-y+z^2)$
\end{enumerate}

\subsubsection*{Actividad 52}
\begin{enumerate}[a)]
\item \begin{enumerate}[a)]
	\item $(x+5)^2$
	\item $(x+10)^2$
	\item $(y+3)^2$
	\item $(\frac{3}{5}x^3+2y)^2$
	\item $(4-x)^2$
	\item No es trinomio cuadrado perfecto
	\item $(6mn^2 -2x^3)^2$
	\end{enumerate}
\item \begin{enumerate}[a)]
	\item $(x-10)(x+4)$
	\item $4(x-1)(x-2)$
	\item No tiene raices en los reales
	\item $2(x-5)(x+3)$
	\item $(x+3)^2$
	\item No tiene raices en los reales
	\end{enumerate}
\end{enumerate}

\subsubsection*{Actividad 53}
\begin{enumerate}[a)]
\item $(5+a)^3$
\item $(2a+3b)^3$
\item $(3x^2+4y^3)^3$
\item $(1+\frac{1}{4}a^2x)^3$
\item $(1-3x)^3$


\end{enumerate}

\subsubsection*{Actividad 54}
\begin{enumerate}[a)]
\item $(2a+3b^2)(2a-3b^2)$
\item $(6m^2+2n)$
\item $(12x+7y^2)(12x-7y^2)$
\item $(\frac{1}{3} a^3 + \frac{1}{5} m^2 d)(\frac{1}{3} a^3 - \frac{1}{5} m^2 d)$
\item $(7c^2+\frac{11}{13}z^3)(7c^2-\frac{11}{13}z^3)$
\item $(0,5m^2+0,3x)(0,5m^2-0,3x)$
\end{enumerate}

\subsubsection*{Actividad 55}
\begin{enumerate}[a)]
\item $(x+5)(x^2 -5x + 25)$
\item $(y+1)(y^6-y^5+y^4-y^3+y^2-y+1)$
\item $(x-\frac{1}{2})(x^2 + \frac{1}{2} x + \frac{1}{4})$
\item $(x-2)(x^4+2x^3+4x^2+8x+16)$
\item $(y+2)(y^6 - 2y^5 + 4y^4 - 8y^3 + 16y^2 - 32y + 64)$
\end{enumerate}

\subsubsection*{Actividad 57}
\begin{enumerate}[a)]
\item $2$
\item $\frac{1}{x+2}$
\item $\frac{x}{x-7}$
\item $\frac{2}{a+x}$
\item $1$
\item $\frac{x}{y-2}$
\end{enumerate}

\subsubsection*{Actividad 58}
\begin{enumerate}[a)]
\item $9$
\item $\frac{2a^2}{x+y}$
\item $\frac{1}{a+1}$
\item $\frac{3x}{2}$
\item $7x$
\item $\frac{-2x^2 + 15}{x+y}$
\end{enumerate}

\subsubsection*{Actividad 59}
\begin{enumerate}[a)]
\item $\frac{6x+16}{(x+4)^2}$
\item $\frac{-x+4}{x+4}$
\item $\frac{2a(a^2+3b^2)}{a^2 - b^2}$
\item $\frac{6x^2+7x+3}{4x^2-1}$
\item $\frac{x}{(x+1)(x-1)}$
\item $\frac{x^2 + 1}{(x+1)^2(x-1)}$
\item $\frac{-x-2}{(x-1)x}$
\item $\frac{3x+1}{(x-1)(x+1)}$ , (para $x \neq -3$)
\item $-\frac{6}{(x-3)(x+3)^2}$
\item $-\frac{2x^3-12x^2-19x+3}{(x-3)(x^2-6x-9)}$ , (para $x \neq 0$)	
\end{enumerate}

\subsubsection*{Actividad 60}
\begin{enumerate}[a)]
\item $\frac{x-7}{2}$
\item $\frac{x}{3}$
\item $\frac{x+2}{4}$
\item $x(x-\frac{2}{3})$
\item $\frac{x+1}{5}$
\item $\frac{2(x^2-4)}{x}$
\item $2-x$
\item $\frac{x-1}{4}$
\end{enumerate}

\subsubsection*{Actividad 61}
\begin{enumerate}[a)]
\item $\frac{1}{3}$
\item $\frac{4(x+1)}{x(x^2+x+1)}$
\item $\frac{x^2 -1}{x^2}$ , (para $x \neq-1 $ y $ x\neq1$) 
\item $-\frac{1}{x}$ , (para $x \neq-1 $ y $x\neq1$)
\item $\frac{2x+13}{(x+2)(x-2)}$
\item $-\frac{4}{3}\frac{x-\frac{1}{6}}{x^2(x-\frac{1}{3})(x-1)}$
\end{enumerate}

\subsubsection*{Actividad 62}
\begin{enumerate}[a)]
\item $m=\frac{7}{5}$
\item $x=\frac{-20}{6}$
\item $x=-6$
\item $z=-\frac{105}{13}$
\item $x=-\frac{13}{4}$
\item $x=\frac{3}{2}$
\item $x=-3$
\item $x=\frac{2}{9}$
\item $x=-\frac{13}{6}$
\item $x=24$
\item $x=-4$
\item $x=5$
\end{enumerate}

\subsubsection*{Actividad 63}
\begin{enumerate}[a)]
\item $10\mathrm{m}$ y $6\mathrm{m}$
\item $35^\circ$, $70^\circ$ y $75^\circ$
\item 900, 1500, 150 y 100
\item 150
\item 21, 14 y 17
\item 191, 382 y 567
\item 9,9
\item 60
\end{enumerate}


% Ya están las respuestas en el enunciado
%\subsubsection*{Actividad 64}
%\begin{enumerate}[a)]
%\item $x=-2$, $y=3$
%\item $x=10$, $y=-4$
%\item $x=-2$, $y=1$
%\end{enumerate}

%\subsubsection*{Actividad 65}
%\begin{enumerate}[a)]
%\item 
%\end{enumerate}


\end{document}
